\documentclass[11pt,a4paper]{report}
\begin{document}

\chapter*{MMseqs user guide}
\renewcommand*\thesection{\arabic{section}}

\textbf{MMseqs suite for fast and sensitive batch searching and clustering
of huge protein sequence sets}

(c) 2015 Maria Hauser, Martin Steinegger, and Johannes Soeding


\section{Summary}

MMseqs (Many-against-Many sequence searching) is a software suite
for very fast protein sequence searches and clustering of huge protein
sequence data sets. MMseqs is around 1000 times faster than protein
BLAST and sensitive enough to capture similarities down to less than
30\% sequence identity.

At the core of MMseqs are two modules for the comparison of two sequence
sets with each other - the prefiltering and the alignment modules.
The first, prefiltering module computes the similarities between all
sequences in one set with all sequences in the other based on a very
fast and sensitive alignment-free metric, the sum of scores of similar
$k$-mers. The alignment module implements an SSE2-accelerated Smith-Waterman-alignment
of all sequences that pass a cut-off for the prefiltering score in
the first module. Both modules are parallelized to use all cores of
a computer to full capacity. Due to its unparalleled combination of
speed and sensitivity, searches of all predicted ORFs in large metagenomics
data sets through the entire UniProtKB or NCBI-NR databases are feasible.
This could allow for assigning to functional clusters and taxonomic
clades many reads that are too diverged to be mappable by current
software.

MMseqs' clustering module can cluster sequence sets efficiently into
groups of similar sequences. It takes as input the similarity graph
obtained from the comparison of the sequence set with itself in the
prefiltering and alignment modules. MMseqs further supports an updating
mode in which sequences can be added to an existing clustering with
stable cluster identifiers and without the need to recluster the entire
sequence set. We will use MMseqs to regularly update versions of the
UniProtKB database clustered down to 20-30\% sequence similarity threshold.


\section{Installation}

First, set environment variables:

\texttt{\$ export MMDIR=\$HOME/path/to/mmseqs/}

\texttt{\$ export PATH=\$PATH:\$MMDIR/bin}

MMseqs uses ffindex, a fast and simple database for wrapping and accessing
huge amounts of small files. Setting the environment variable \texttt{LD\_LIBRARY\_PATH
}ensures that ffindex binaries are in the path:

\texttt{\$ export LD\_LIBRARY\_PATH = \$LD\_LIBRARY\_PATH:\$MMDIR/lib/ffindex/src}

Then create MMseqs binaries:

\texttt{\$ cd \$MMDIR/src}

\texttt{\$ make}

MMseqs binaries are now located in \texttt{\$MMDIR/bin.}


\section{Getting started}

Here we explain how to run a search for matches of sequences in the
query database in the target database and how to cluster a database.
Test data (a query and a target database for the sequence search and
a database for the clustering) are stored in \texttt{\$MMDIR/data.}


\subsection*{Search}

You can use the query database \texttt{queryDB.fasta} and target database
\texttt{targetDB.fasta} to test the search workflow.

Before clustering, you need to convert your database containing query
sequences (\texttt{queryDB.fasta}) and your target database (\texttt{targetDB.fasta})
into ffindex format:

\texttt{\$ mmseqs createdb queryDB.fasta queryDB}

\texttt{\$ mmseqs createdb targetDB.fasta targetDB}
 
It generates ffindex database files, e. g. \texttt{queryDB} and ffindex
index file \texttt{queryDB.index} from \texttt{queryDB.fasta}.

For the next step (the prefiltering) the \texttt{targetDB} has to be prepared for fast read in:

\texttt{\$ mmseqs createindex targetDB targetDB.fastindex}

 Then,generate a directory for tmp files:

\texttt{\$ mkdir tmp}

Please ensure that in case of large input databases \texttt{tmp} provides
enough free space. For the disc space requirements, see the section

The alignment consists of two steps the \texttt{prefilter} and \texttt{alignment}.
To run the search, type:

\texttt{\$ mmseqs prefilter queryDB targetDB.fastindex outDB.prefilter }

\texttt{\$ mmseqs alignment queryDB targetDB outDB.prefilter outDB }

Then, convert the result ffindex database into a FASTA formatted database:

\texttt{\$ mmseqs createfasta outDB outDB.fasta}


\subsection*{Clustering}

Before clustering, convert your FASTA database into ffindex format:

\texttt{\$ mmseqs createdb DB.fasta DB}

Then, generate a directory for tmp files:

\texttt{\$ mkdir tmp}

Run the cascaded clustering of your database and output the result
into the database files \texttt{DB\_clu}, \texttt{DB\_clu.index}:

\texttt{\$ mmseqs clusteringworkflow DB DB\_clu tmp}

To generate a FASTA-style formatted output file from the ffindex output
file, type:

\texttt{\$ mmseqs createfasta DB\_clu DB\_clu.fasta}

To run the more sensitive cascaded clustering and convert the result
into FASTA format, type:

\texttt{\$ mmseqs clusteringworkflow DB DB\_clu\_s7 tmp --cascaded -s 7}

\texttt{\$ mmseqs createfasta DB\_clu\_s7 DB\_clu\_s7.fasta}


\section{System requirements}

MMseqs runs under Linux only. Alignment and prefiltering modules are
fully parallelized with SSE2 and OpenMP, i. e. MMseqs runs fastest
on a computer with many (e.g. 16-32) cores. Besides, MMseqs needs
much memory (you need 128G of memory in order to cluster the current
UniProtKB version containing 54\,M sequences). We offer an option
for limiting the memory use at the cost of longer runtimes. The database
is split into chunks and the program only holds one chunk in memory
at any time. For clustering large databases containing tens of millions
of sequences, you should provide enough free disc space ($\approx$500
GB). In section \ref{sec:Sensitivity-and-consumption}, we will discuss
the runtime, memory and disc space consumption of MMseqs and how to
reduce resource requirements for large databases.


\section{ffindex database format}

All modules take ffindex databases as input and produce ffindex databases
as output. ffindex was developed to avoid drastically slowing down
the file system when millions of files need to be written and accessed.
ffindex hides the files from the file system by storing them as unstructured
data records in a single \emph{data file}. In addition to this data
file, an ffindex database includes a second file: This \emph{index
file }stores for each unique accession code the start position in
bytes of the data record in the ffindex data file. When transforming
a FASTA file with multiple sequences into an ffindex database, the
accession code is the ID of the sequence parsed from the header. If
no ID can be identified, the accession code is the whole header without
the > character before the first blank space.

The binaries \texttt{fasta2ffindex} and \texttt{ffindex2fasta} located
in \texttt{mmseqs/bin} do the format conversion from and to the ffindex
database format. \texttt{fasta2ffindex} generates a ffindex database
from a FASTA sequence database. \texttt{ffindex2fasta} converts an
ffindex database to a FASTA formatted text file: the headers are ffindex
accession codes preceded by >, with the corresponding dataset from
the ffindex data file following.

However, for a fast access to the particular datasets in very large
databases it is advisable to use the ffindex database directly without
converting. We offer the binary \texttt{ffindex\_get} for direct access
to the datasets stored in an ffindex database.


\section{Overview of folders in MMseqs}
\begin{itemize}
\item \texttt{bin}: MMseqs binaries, \texttt{fasta2ffindex} and \texttt{ffindex2fasta
}binaries
\item \texttt{data}: BLOSUM matrices and test data
\item \texttt{lib}: ffindex sources and binaries
\item \texttt{src}: MMseqs sources and the Makefile
\end{itemize}

\section{Overview of MMseqs commands}

MMseqs contains five modules. Three commands execute workflows that
combine MMseqs core modules. The other three commands execute the
single modules which are used by the workflows and should be used
by more advanced users.

Workflows:
\begin{itemize}
\item \texttt{mmseqs search:} Compares all sequences in the query database
with all sequences in the target database running prefiltering and alignment module.
\item \texttt{mmseqs clusteringworkflow:} Clusters the sequences in the input database
by sequence similarity.
\item \texttt{mmseqs clusterupdate:} Given an the existing clustering of a sequence
database and a new version of the sequence database with some new
sequences being added and others having been deleted, MMseqs incrementally
updates the clustering.
\end{itemize}
Single modules:
\begin{itemize}
\item \texttt{mmseqs prefilter:} Computes k-mer similarity scores between all
sequences in the query database and all sequences in the target database.
\item \texttt{mmseqs alignment:} Computes Smith-Waterman alignment scores between
all sequences in the query database and the sequences of the target
database whose prefiltering scores computed by \texttt{mmseqs prefilter}
pass a minimum threshold.
\item \texttt{mmseqs clustering:} Computes a similarity clustering of a sequence
database based on Smith Waterman alignment scores of the sequence
pairs computed by \texttt{mmseqs alignment}.
\end{itemize}

\section{Description of workflows}


\subsection{Batch sequence searching using \texttt{mmseqs search}}

For searching a database, you need your query and target database
in ffindex format and an empty directory for MMseqs temporary files.
Then, you can run the search by typing

\texttt{\$ mmseqs search queryDB targetDB outDB tmp}

To get more sensitive results, increase the search sensitivity \texttt{(-s}
option):

\texttt{\$ mmseqs search queryDB targetDB outDB tmp -s 7}

The default sensitivity is 4, sensitivity can be set in the range
$[1:9]$.

This workflow combines the prefiltering and alignment modules into
a fast and sensitive batch protein sequence search that compares all
sequences in the query database with all sequences in the target database\emph{.
}Query and target databases may be identical. The program outputs
for each query sequence all database sequences satisfying the search
criteria such as sensitivity of the search.

The underlying algorithm is explained in more detail in section \ref{sub:Prefiltering},
and the full parameter list can be found in section \ref{sub:Search-workflow}.


\subsection{Clustering databases using \texttt{mmseqs clusteringworkflow}\label{sub:Clustering}}

For clustering a database, your need your sequence database in ffindex
format and an empty directory for MMseqs temporary files. Then, you
can run the clustering with

\texttt{\$ mmseqs clusteringworkflow inDB outDB tmp}

and cascaded clustering with

\texttt{\$ mmseqs clusteringworkflow inDB outDB tmp -{}-cascaded}

For more sensitive clustering, adjust the sensitivity (\texttt{-s}
option):

\texttt{\$ mmseqs clusteringworkflow inDB outDB tmp -{}-cascaded -s 7}

The clustering workflow combines the prefiltering, alignment and clustering
modules into a simple clustering or\emph{ }cascaded clustering of
a database. There are two possibilities to run the clustering:
\begin{itemize}
\item Simple clustering\emph{ }runs the prefiltering, alignment and clustering
modules with predefined parameters.
\item Cascaded clustering\emph{ }clusters the sequence database using prefiltering,
alignment and clustering modules incrementally in three steps. In
the first step, the prefiltering runs with a low sensitivity of 1
and a very high results significance threshold in order to accelerate
the calculation and search only for hits with a high sequence identity.
Then alignments are calculated and the database is clustered. The
second step takes the representative sequences of the first clustering
step and repeats the prefiltering, alignment and clustering steps.
This time, the prefiltering is run with a higher sensitivity and a
lower result significance threshold for catching sequence pairs with
lower sequence identity. In the third step, the whole process is repeated
again with the target sensitivity defined by the \texttt{-s} parameter.
Eventually, the clustering results are merged and the resulting clustering
is written to the output ffindex database. 
\end{itemize}
Cascaded clustering yields more sensitive results than simple clustering.
Also, it allows very large cluster sizes in the end clustering resulting
from cluster merging (note that cluster size can grow exponentially
in the cascaded clustering workflow), which is not possible with the
simple clustering workflow because of the limited maximum number of
sequences passing the prefiltering and the alignment. Therefore, we
strongly recommend to use cascaded clustering especially to cluster
larger databases and to obtain maximum sensitivity.

The underlying algorithm is explained in more detail in section \ref{sub:Prefiltering},
and the full parameter list can be found in section \ref{sub:Clustering-workflow}.


\subsection{Updating a database clustering using \texttt{mmseqs clusterupdate}}

To run the updating, you need the old and the new version of your
sequence database in ffindex format, the clustering of the old database
version and a directory for the temporary files:

\texttt{\$ mmseqs clusterupdate oldDB newDB oldDB\_clustering outDB tmp}

This workflow efficiently updates the clustering of a database by
adding new and removing outdated sequences. It takes as input the
older sequence database, the results obtained by this older database
clustering, and the newer version of the sequence database. Then it
adds the new sequences to the clustering and removes the sequences
that were removed from the newer database. Sequences which are not
similar enough to any existing cluster will be founders of new clusters.


\section{Description of core modules}

For advanced users, it is possible to run core modules for maximum
flexibility. Especially for the sequence search it can be useful to
adjust the prefiltering and alignment parameters according to the
needs of the user. The detailed parameter lists for the modules is
provided in section \ref{sec:Detailed-parameter-list}.

MMseqs contains\textbf{ }three core modules: prefiltering, alignment
and clustering.


\subsection{Computation of prefiltering scores using \texttt{mmseqs prefilter}\label{sub:Prefiltering}}

The prefiltering module calculates the sum of scores of similar $k$-mers
between all query sequences and all database sequences and outputs
the most similar sequence pairs. 

If you want to \emph{cluster} a database, or do an all-against-all
search, you only have one input database. In this case, the prefiltering
does an all-against-all search with the following program call:

\texttt{\$ mmseqs prefilter inputDB inputDB resultDB\_pref }

\texttt{inputDB} is the base name of the ffindex databases you produced
from your FASTA sequence databases, the prefiltering results are stored
in the ffindex database files \texttt{resultDB\_pref}, \texttt{resultDB\_pref.index}.

If you want to do a \emph{sequence search}, you have two input databases:
query database and target database, that also can be identical. In
this case, the prefiltering program call is:

\texttt{\$ mmseqs prefilter queryDB targetDB resultDB\_pref }

First, the database sequences are indexed in an index table to provide
a fast access to the $k$-mers of the database sequences. The index
table has an array with a pointer for each possible $k$-mer to an
index list storing the IDs of the database sequences containing this
$k$-mer. After the index table generation, the algorithm processes
each query sequence from left to right and generates a list of similar
$k$-mers for the $k$-mer at the current query sequence position.
For each $k$-mer in the list, the $k$-mer similarity score is added
to the overall prefiltering score for each database sequence containing
this $k$-mer, retrieved using the index table. After processing the
whole query sequence, database sequences with significant prefiltering
scores are extracted and written to the prefiltering result database. 

Since different queries yield different score distribution in the
database, a rigid prefiltering score threshold does not work well.
The statistical significance of a prefiltering score for a given query
sequence is described by the Z-score. The Z-score for a prefiltering
score of a query sequence and a database sequence is calculated based
on the score distribution for the query in the database. For each
query and database sequence pair, the prefiltering score $S_{qt}$
is normalized by subtracting the background score $S_{0}$ expected
by chance and dividing by the standard deviation of the score $\sigma_{S}$,
resulting in a normalized Z-score $Z_{qt}$:

\begin{eqnarray*}
Z_{qt} & = & \frac{S_{qt}-S_{0}}{\sigma_{S}}\end{eqnarray*}


Instead of setting a prefiltering score threshold, we set a rigid
Z-score (i. e. result significance) threshold. Only results with a
sufficient Z-score are written to the output.

The sensitivity of the prefiltering can be set using the \texttt{-s}
option. Internally, \texttt{-s} sets the average length of the lists
of similar $k$-mers per query sequence position and the Z-score threshold. 
\begin{itemize}
\item \emph{Similar $k$-mers list length:} Low sensitivity yields short
similar $k$-mer lists. Therefore, the speed of the prefiltering increases,
since only short $k$-mer lists have to be generated and less lookups
in the index table are necessary. However, the sensitivity of the
search decreases, since only very similar $k$-mers are generated
and therefore, the prefiltering can not identify sequence pairs with
low sequence identity.
\item \emph{Z-score threshold:}\textbf{ }Z-score of a prefiltering result
describes its statistical significance. Lower sensitivity yields a
higher Z-score threshold, i. e. only the most significant results
are displayed.
\end{itemize}
Furthermore, there is a possibility to use different lengths of the
$k$-mers used in the prefiltering. Longer $k$-mers are more sensitive,
since they cause less chance matches. On the other hand, for a fixed
run time, longer $k$-mers only pay off for larger databases. The
reason is the different relation between the time for the $k$-mer
list generation and database matching for different database sizes.
Longer $k$-mers need more time for the $k$-mer list generation,
but less time for database matching. Therefore, the database matching
should take most of the computation time, which is only the case for
large databases. For a fixed run time, the default value $k=6$ is
the best for databases containing a few million sequences. For very
large databases containing about 100 million sequences, $k=7$ should
be a better choice theoretically, though the real life performance
of $7$-mers on large databases is not tested yet. For databases containing
only hundreds of thousands of sequences, $k=5$ should be sufficient. 


\subsection{Local alignment of prefiltering sequences using \texttt{mmseqs alignment}}

In the alignment module, you can also specify either identical or
different query and target databases. If you want to do a clustering
in the next step, query and target database need to be identical.

\texttt{\$ mmseqs alignment inputDB inputDB resultDB\_pref resultDB\_aln }

Alignment results are stored in the ffindex files \texttt{resultDB\_aln},
\texttt{resultDB\_aln.index}.

Program call in case you want to do the sequence search and have different
query and target databases:

\texttt{\$ mmseqs alignment queryDB targetDB resultDB\_pref resultDB\_aln }

This module implements an SSE2-accelerated Smith-Waterman-alignment
(Farrar, 2007) of all sequences that pass a cut-off for the prefiltering
score in the first module. It processes each sequence pair from the
prefiltering results and aligns them in parallel, calculating one
alignment per core at a single point of time. Additionally, the alignment
calculation is vectorized using SIMD (single instruction multiple
data) instructions. Eventually, the alignment module calculates alignment
statistics such as sequence identity, alignment coverage and e-value
of the alignment.


\subsection{Clustering sequence database using \texttt{mmseqs cluster}\label{sub:Clustering_module}}

For the clustering, you need the input sequence database and the alignment
results for the database:

\texttt{\$ mmseqs cluster inputDB resultsDB\_aln resultsDB\_clu}

Clustering results are stored in the ffindex database files \texttt{resultsDB\_clu},
\texttt{resultsDB\_clu.index}.

The clustering module offers the possibility to run three different
clustering algorithms. A greedy set cover algorithm is the default.
It tries to cover the database by as few clusters as possible. At
each step, it forms a cluster containing the representative sequence
with the most alignments above the special or default thresholds with
other sequences of the database and these matched sequences. Then,
the sequences contained in the cluster are removed and the next representative
sequence is chosen.

The second clustering algorithm is a greedy clustering algorithm, as
used in CD-HIT. It sorts sequences by length and in each step forms
a cluster containing the longest sequence and sequences that it matches.
Then, these sequences are removed and the next cluster is chosen from
the remaining sequences.

The third clustering algorithm is affinity propagation algorithm. This algorithm is based on passing messages of two types between sequences: 1. how likely is the sequence part of the others sequence cluster. 2. how likely is the sequence a representative. 

Note that we \emph{always} recommend to use the cascaded clustering
workflow instead of the clustering module for larger databases, since
the maximum cluster size is limited to a quite low value otherwise
(between 50 and 300 for large databases containing millions of sequences,
depending on the database size). The reasons are the limited result
list length in the prefiltering and alignment modules (the maximum
list length determines the maximum cluster size in the simple clustering
workflow) and the high memory consumption of the clustering for large
databases with many alignment results per query.


\section{Output file formats}

Results of MMseqs commands are stored in ffindex databases. All records
within those ffindex databases are in plain text format. 


\subsection{Prefiltering}

The ffindex accession code is the UniProtKB ID (or other ID depending
on the database format) of the query. A line in the prefiltering result
database record (= one match) has the following format:

\texttt{targetId Z-score prefilteringScore}

where \texttt{targetId} is the database ID of the matched sequence,
\texttt{Z-score} is the statistical significance score of the match
and \texttt{prefilteringScore} is the raw score of the match (the
sum of the scores of similar $k$-mers of the query and target sequence)
in half bits. Example of a prefiltering result for the SwissProt sequence
Q54G30 (excerpt):

\texttt{\footnotesize Q54G30 1177.95 55735}{\footnotesize \par}

\texttt{\footnotesize Q869W0 159.179 5274}{\footnotesize \par}

\texttt{\footnotesize Q86IM3 99.2044 1823}{\footnotesize \par}

\texttt{\footnotesize Q54E43 85.8743 3224}{\footnotesize \par}

The first match is the identity Q54G30 having a very high prefiltering
score of 55735 and the Z-score of 1177.95. 


\subsection{Alignment}

The ffindex accession code is the UniProtKB ID (or other ID depending
on the database format) of the query. One line of the alignment results
record has the following format:

\texttt{targetId alnScore queryCov targetCov seqId evalue}

where \texttt{targetId} is the database ID of the matched sequence,
\texttt{alnScore} is the raw score of the alignment in half bits,
\texttt{queryCov} is the alignment coverage of the query in the range
$[0:1]$, \texttt{targetCov} is the alignment coverage of the target
database sequence in the range $[0:1]$, \texttt{seqId} is the sequence
identity and \texttt{evalue} is the e-value of the match. Example
of an alignment result for the SwissProt sequence A0PUH6 (excerpt):

\texttt{\footnotesize A0PUH6 1305 1.000 1.000 1.000 1.507e-186}{\footnotesize \par}

\texttt{\footnotesize Q6NFN4 824 0.956 0.974 0.649 3.682e-114}{\footnotesize \par}

\texttt{\footnotesize Q8DD39 256 0.900 0.909 0.335 1.136e-28}{\footnotesize \par}

\texttt{\footnotesize P52973 182 0.808 0.822 0.238 1.597e-17}{\footnotesize \par}

The first line is the identity match. The last sequence P52973 has
a Smith-Waterman alignment score 182, query sequence coverage 0.808,
database sequence coverage 0.822, the alignment has the sequence identity
0.238 and the e-value 1.597e-17.


\subsection{Clustering}

Every cluster is stored once (i. e. one result database record per
cluster). Each database record contains UniProtKB IDs (or other IDs
depending on the database format) of the sequences assigned to this
cluster, one ID per line. The ffindex accession code is the ID of
the representative sequence of the cluster. An example of a cluster
record with 4 cluster members:

\texttt{\footnotesize Q9ZZZ1 }{\footnotesize \par}

\texttt{\footnotesize Q96189 }{\footnotesize \par}

\texttt{\footnotesize O03850 }{\footnotesize \par}

\texttt{\footnotesize P03887 }{\footnotesize \par}


\section{Optimizing sensitivity and consumption of resources\label{sec:Sensitivity-and-consumption}}

This section discusses how to keep the run time, the memory and disc
space consumption of MMseqs at reasonable values, while obtaining
results with the highest possible sensitivity. These considerations
are relevant if the size of your database exceeds several millions
of sequences and they are most relevant if the database size is in
the order of tens of millions of sequences.


\subsection{Prefiltering module}

The prefiltering module can use a lot of resources regarding all the
memory consumption, the runtime and the disc space, if the parameters
are not set appropriately.


\paragraph{Memory consumption}

For maximum efficiency of the prefiltering, the entire database should
be held in RAM memory. The major part of memory is required for the
$k$-mer index table of the database. For a database containing $N$
sequences with an average length $L$, the memory consumption of the
index lists is $N\times L\times4$B. Note that the memory consumption
grows linearly with the size of the sequence database. In addition,
the index table stores the pointer array and two auxiliary arrays
with the memory consumption of $a^{k}\times16$B, where $a$ is the
size of the amino acid alphabet (usually 21 including the unknown
amino acid X) and $k$ is the $k$-mer size. The overall memory consumption
of the index table is

\begin{eqnarray*}
M & = & (4\, N\, L+16\, a^{k})\mathrm{B}\end{eqnarray*}
 

Therefore, the UniProtKB database version of April 2014 containing
55 million sequences with an average length 350 needs about 71 GB
of main memory. 

To limit the memory use at the cost of longer runtimes, the option
\texttt{-{}-max-chunk-size} allows the user to split the database
into chunks of the given maximum size. 


\paragraph{Runtime}

The prefiltering module is the most time consuming step. To cluster
the 55 million sequences of UniProtKB (04/2014), the MMseqs prefiltering
module needs about 6 days when running on 32 cores and about 10 days
when running on 16 cores of a modern computer.


\paragraph{Disc space}

The prefiltering results for very large databases can grow to considerable
sizes (in the order of TB) of the disc space if very long result lists
are allowed and a low Z-score threshold is set. As an example, an
all-against-all prefiltering run on the UniProtKB with \texttt{-{}-max-seqs}
300 yielded average prefiltering list length 150 and the output file
size 146 GB.


\paragraph{Important options for tuning the memory, runtime and disc space usage}
\begin{itemize}
\item The option \texttt{-s} controls the sensitivity in the MMseqs prefiltering
module. The lower the sensitivity, the faster the prefiltering gets
at the cost of search sensitivity. Default sensitivity is 4, increasing
the sensitivity by one roughly doubles the runtime of the prefiltering.
In order to cluster the UniProtKB down to $\approx$30\% sequence
identity, you should leave this parameter at the default value of
4. For clustering down to 90\%, sensitivity 1 should be sufficient,
although there are still no specific tests for the optimum parameters
necessary for clustering down to a fixed sequence identity.
\item The option \texttt{-{}-max-seqs }controls the maximum number of prefiltering
results per query sequence. For very large databases (tens of millions
of sequences), it is a good advice to keep this number at reasonable
values (i. e. the default value 300). For considerably larger values
of \texttt{-{}-max-seqs}, the size of the output can be in the range
of several TB of disc space for databases containing tens of millions
of sequences. Changing \texttt{-{}-max-seqs} option has no effect
on the run time.
\item The option \texttt{-{}-z-score} describes the minimum significance
of the results written to the output. Usually, this option is set
automatically depending on the sensitivity. However, especially for
the sequence search it can be desired to see also less significant
results. Setting \texttt{-{}-z-score} at lower values yields more
results and therefore increases the size of the output written to
disc. In addition, it slows down the program.
\end{itemize}

\subsection{Alignment module}

In the alignment module, generally only the runtime is a critical
issue.


\paragraph{Memory consumption}

The major part of the memory is required for the three dynamic programming
matrices, once per core. Since most sequences are quite short, the
memory requirements of the alignment module for a typical database
are in the order of a few GB. 


\paragraph{Runtime}

It takes about 2-3 days to compute Smith-Waterman alignments for the
UniProtKB sequence pairs which passed the prefiltering step (at default
parameters for deep clustering down to $\approx$20 - 30\% pairwise
sequence identity).

If a huge amount of alignments has to be calculated, the run time
of the alignment module can become a bottleneck. The run time of the
alignment module depends essentially on two parameters:
\begin{itemize}
\item The option \texttt{-{}-max-seqs }controls the maximum number of sequences
aligned with a query sequence. By setting this parameter to a lower
value, you accelerate the program, but you may also lose some meaningful
results. Since the prefiltering results are always ordered by their
significance, the most significant prefiltering results are always
aligned first in the alignment module.
\item The option \texttt{-{}-max-rejected} defines the maximum number of
rejected sequences for a query until the calculation of alignments
stops. A reject is an alignment whose statistics don't satisfy the
search criteria such as coverage threshold, e-value threshold etc.
Per default, \texttt{-{}-max-rejected} is set to \texttt{INT\_MAX},
i. e. all alignments until \texttt{-{}-max-seqs} alignments are calculated.
\end{itemize}

\paragraph{Disc space}

Since the alignment module takes the results of the prefiltering module
as input, the size of the prefiltering module output is the point
of reference. If alignments are calculated and written for all the
prefiltering results, the disc space consumption is $1.75$ times
higher than the prefiltering output size.


\subsection{Clustering module}

In the clustering module, only the memory consumption is a critical
issue.


\paragraph{Memory consumption}

The clustering module can need large amounts of memory. The memory
consumption for a database containing $N$ sequences and an average
of $r$ alignment results per sequence can be estimated as

\begin{eqnarray*}
M & = & 40\times N\times r\,\mathrm{B}\end{eqnarray*}
 

To prevent excessive memory usage for the clustering of large databases,
you should use cascaded clustering (\texttt{-{}-cascaded} option)
which accumulates sequences per cluster incrementally, therefore avoiding
excessive memory use.

If you run the clustering module separately, you can tune
\begin{itemize}
\item \texttt{-{}-max-seqs} parameter which controls the maximum number
of alignment results per query considered (i. e. the number of edges
per node in the graph). Lower value causes lower memory usage and
faster run times.
\item Alternatively, \texttt{-s} parameter can be set to a higher value
in order to cluster the database down to higher sequence identities.
Only the alignment results above the sequence identity threshold are
imported and it results in lower memory usage. 
\end{itemize}

\paragraph{Runtime}

Clustering is the fastest step. It needs about 2 hours for the clustering
of the whole UniProtKB. 


\paragraph{Disc space}

Since only one record is written per cluster, the memory usage is
a small fraction of the memory usage in the prefiltering and alignment
modules.


\subsection{Workflows}

The search and clustering workflows offer the possibility to set the
sensitivity option \texttt{-s} and the maximum sequences per query
option \texttt{-{}-max-seqs}. \texttt{-{}-max-rejected} option is
set to \texttt{INT\_MAX} per default. Cascaded clustering sets all
the options controlling the size of the output, speed and memory consumption,
internally adjusting parameters in each cascaded clustering step.


\section{Detailed parameter list\label{sec:Detailed-parameter-list}}


\subsection{Search workflow\label{sub:Search-workflow}}

Compares all sequences in the query database with all sequences in
the target database. 

\textbf{Usage:}

\texttt{mmseqs search <queryDB> <targetDB> <outDB> <tmpDir> {[}opts{]}}

\textbf{Options:}

\texttt{\small -s {[}float{]} Target sensitivity in the range {[}1:9{]}
(default=4).}{\small \par}

Adjusts the sensitivity of the prefiltering and influences the prefiltering
run time. For detailed explanation see section \ref{sub:Prefiltering}.

\texttt{\small -{}-z-score {[}float{]} Z-score threshold (default: 50.0)}{\small \par}

Prefiltering Z-score cutoff. A lower z-score cutoff yields more results,
since also less significant results are written to the output. For
detailed explanation see section \ref{sub:Prefiltering}.

\texttt{\small -{}-max-seqs Maximum result sequences per query (default=300)}{\small \par}

Maximum number of sequences passing the prefiltering and alignment
per query. If the prefiltering result list exceeds the \texttt{-{}-max-seqs}
value, only the sequences with the best Z-score pass the prefiltering
and are aligned in the alignment step.

\texttt{\small -{}-max-seq-len {[}int{]} Maximum sequence length (default=50000).}{\small \par}

The length of the longest sequence in the input database.

\texttt{\small -{}-sub-mat {[}file{]} Amino acid substitution matrix
file (default: BLOSUM62).}{\small \par}

Substitution matrices for different sequence diversities in the required
format can be found in the MMseqs \texttt{data} folder.


\subsection{Clustering workflow\label{sub:Clustering-workflow}}

Calculates the clustering of the sequences in the input database. 

\textbf{Usage:}

\texttt{mmseqs clusteringworkflow <sequenceDB> <outDB> <tmpDir> {[}opts{]}}

\textbf{Options:}

\texttt{\small -{}-cascaded Start the cascaded instead of simple clustering
workflow.}{\small \par}

The database is clustered incrementally in three steps and improves
the sensitivity of the clustering greatly compared to the general
workflow. For detailed explanation, see the section \ref{sub:Clustering}.

\texttt{\small -s {[}float{]} Target sensitivity in the range {[}2:9{]}
(default=4).}{\small \par}

Adjusts the sensitivity of the prefiltering and influences the prefiltering
run time. For detailed explanation see section \ref{sub:Prefiltering}.

\texttt{\small -{}-max-seqs Maximum result sequences per query (default=300).}{\small \par}

Maximum number of sequences passing the prefiltering and alignment
per query. If the prefiltering result list exceeds the \texttt{-{}-max-seqs}
value, only the sequences with the best Z-score pass the prefiltering
and are aligned in the alignment step.

\texttt{\small -{}-max-seq-len {[}int{]} Maximum sequence length (default=50000).}{\small \par}

The length of the longest sequence in the database.

\texttt{\small -{}-sub-mat {[}file{]} Amino acid substitution matrix
file.}{\small \par}

Substitution matrices for different sequence diversities in the required
format can be found in the MMseqs \texttt{data} folder.


\subsection{Updating workflow}

Updates the existing clustering of the previous database version with
new sequences from the current version of the same database.

\textbf{Usage:}

\texttt{mmseqs clusterupdate <oldDB> <newDB> <oldDB\_clustering> <outDB>
<tmpDir> {[}opts{]}}

\textbf{Options:}

\texttt{\small -{}-sub-mat {[}file{]} Amino acid substitution matrix
file.}{\small \par}

Substitution matrices for different sequence diversities in the required
format can be found in the MMseqs \texttt{data} folder.

\texttt{\small -{}-max-seq-len {[}int{]} Maximum sequence length (default=50000).}{\small \par}

The length of the longest sequence in the database.


\subsection{Prefiltering}

Calculates k-mer similarity scores between all sequences in the query
database and all sequences in the target database.

\textbf{Usage:}

\texttt{mmseqs prefilter <queryDB> <targetDB> <outDB> {[}opts{]}}

\textbf{Options:}

\texttt{\small -s {[}float{]} Sensitivity in the range {[}1:9{]} (default=4).}{\small \par}

Adjusts the sensitivity of the prefiltering and influences the prefiltering
run time. For detailed explanation see section \ref{sub:Prefiltering}.

\texttt{\small -k {[}int{]} k-mer size in the range {[}4:7{]} (default=6).}{\small \par}

The size of $k$-mers used in the prefiltering. For guidelines for
choosing a different $k$ as the default, see section \ref{sub:Prefiltering}.

\texttt{\small --k-score {[}int{]}  Set the K-mer threshold for the K-mer generation.}{\small \par}


\texttt{\small -{}-alph-size {[}int{]} Amino acid alphabet size (default=21).}{\small \par}

Amino acid alphabet size, default = 21 (full amino acid alphabet).
For using a reduced amino acid alphabet, choose a lower value. Reduced
amino acid alphabets reduce the memory usage, but also the sensitivity.

\texttt{\small -{}-max-seq-len {[}int{]} Maximum sequence length (default=50000).}{\small \par}

The length of the longest sequence in the database.

\texttt{\small --profile HMM Profile input.}{\small \par}


\texttt{\small --nucl Nucleotide sequences input.}{\small \par}

\texttt{\small -{}-z-score {[}float{]} Z-score threshold (default: 50.0).}{\small \par}

Prefiltering Z-score cutoff. A lower z-score cutoff yields more results,
since also less significant results are written to the output. For
detailed explanation see section \ref{sub:Prefiltering}.

\texttt{\small -{}-skip {[}int{]} Number of skipped k-mers during
the index table generation.}{\small \par}

Number of $k$-mers in the database sequences skipped during the index
table generation. Per default, each $k$-mer of the database is indexed.
With $\mathrm{skip}=2$, $2$ $k$-mers are skipped and only each
third $k$-mer is indexed. This speeds up the search and reduces the
memory usage at the cost of lower search sensitivity.

\texttt{\small -{}-max-seqs {[}int{]} Maximum result sequences per
query (default=300).}{\small \par}

Maximum number of sequences passing the prefiltering per query. If
the prefiltering result list exceeds the \texttt{-{}-max-seqs} value,
only the sequences with the best Z-score pass the prefiltering.

\texttt{\small --search-mode {[}int{]}  Search mode. Global: 0 Local: 1 Local fast: 2.}{\small \par}

\texttt{\small -{}-no-comp-bias-corr Switch off local amino acid composition
bias correction.}{\small \par}

Compositional bias correction assigns lower scores to amino acid matches
of the amino acids that are frequent in their neighborhood in the
query sequence.

\texttt{\small -{}-max-chunk-size {[}int{]} Splits target databases
in chunks when the database size exceeds the given size. (For memory
saving only)}{\small \par}

Maximum number of sequences stored in the index table at some point
of time, default = \texttt{INT\_MAX}. Restraining the number of sequences
stored reduces the memory usage, but slows down the calculation.

\texttt{\small --fast-mode Fast search is using Z-score instead of logP-Value and extracts hits with a score higher than 6}{\small \par}
 


\texttt{\small --spaced-kmer-mode       Spaced kmers mode (use consecutive pattern). Disable: 0             , Enable: 1}{\small \par}


\texttt{\small -{}-sub-mat {[}file{]} Amino acid substitution matrix
file.}{\small \par}

Substitution matrices for different sequence diversities in the required
format can be found in the MMseqs \texttt{data} folder.

\texttt{\small -v {[}int{]} Verbosity level: 0=NOTHING, 1=ERROR, 2=WARNING,
3=INFO (default=3).}{\small \par}

Verbosity level in the range $[0:3]$. With verbosity $0$, there
is no terminal output.
\texttt{\small --threads {[}int{]} Number of cores used for the computation
(default=all cores).}{\small \par}


\subsection{Alignment}

Calculates Smith-Waterman alignment scores between all sequences in
the query database and the sequences of the target database which
passed the prefiltering.

\textbf{Usage:}

\texttt{mmseqs alignment <queryDB> <targetDB> <prefResultsDB> <outDB>
{[}opts{]}}

\textbf{Options:}

\texttt{\small -e {[}float{]} Maximum e-value (default=0.01).}{\small \par}

E-value of the local alignment is calculated using Karlin-Altschul
statistics.

\texttt{\small -c {[}float{]} Minimum alignment coverage (default=0.8).}{\small \par}

Minimum alignment coverage of both query and database sequence, default
= $0.8$. With the value of $0.0$, the alignments are assessed using
only the e-value criterion.

\texttt{\small --min-seq-id Minimum sequence identity of sequences}{\small \par}


\texttt{\small -{}-max-seq-len {[}int{]} Maximum sequence length (default=50000).}{\small \par}

The length of the longest sequence in the database.

\texttt{\small -{}-max-seqs {[}int{]} Maximum alignment results per
query sequence (default=300).}{\small \par}

Maximum number of sequences passing the alignment per query. Sequences
are read in the order of the prefiltering lists. The reading for a
query is stopped if the number of sequences for a query sequence exceeds
the \texttt{-{}-max-seqs} value.

\texttt{\small -{}-max-rejected {[}int{]} Maximum rejected alignments
before alignment calculation for a query is aborted. (default=INT\_MAX)}{\small \par}

Maximum number of rejected alignments for a query until the alignment
calculation is stopped. A rejected alignment is an alignment that
does not satisfy the e-value and alignment coverage thresholds. Default
= \texttt{INT\_MAX} (i. e., all alignments are calculated).

\texttt{\small -{}-nucl Nucleotide sequences input.}{\small \par}

\texttt{\small --profile HMM Profile input.}{\small \par}


\texttt{\small -{}-sub-mat {[}file{]} Amino acid substitution matrix
file.}{\small \par}

Substitution matrices for different sequence diversities in the required
format can be found in the MMseqs \texttt{data} folder.

\texttt{\small --threads {[}int{]} Number of cores used for the computation
(default=all cores).}{\small \par}

\texttt{\small -v {[}int{]} Verbosity level: 0=NOTHING, 1=ERROR, 2=WARNING,
3=INFO (default=3).}{\small \par}

Verbosity level in the range $[0:3]$. With verbosity $0$, there
is no terminal output.


\subsection{Clustering}

Calculates a clustering of a sequence database based on Smith Waterman
alignment scores of the sequence pairs.

\textbf{Usage:}

\texttt{mmseqs cluster <sequenceDB> <alnResultsDB> <outDB> {[}opts{]}}

\textbf{Options:}

\texttt{\small --cluster-mode 0 Setcover, 1 affinity clustering, 2 Greedy clustering by sequence length
).}{\small \par}

For the description of the three algorithms, see section \ref{sub:Clustering_module}.

\texttt{\small --min-seq-id {[}float{]} Minimum sequence identity of sequences
in a cluster (default = 0.0)}{\small \par}

Minimum sequence identity of the cluster members and the representative
sequence. Per default, the sequence identity criterion is switched
off.

\texttt{\small -{}-max-seqs {[}int{]} Maximum result sequences per
query (default=100)}{\small \par}

Maximum alignment results read per query. This is at the same time
the maximum possible number of sequences in the cluster.

\texttt{\small -v {[}int{]} Verbosity level: 0=NOTHING, 1=ERROR, 2=WARNING,
3=INFO (default=3).}{\small \par}

Verbosity level in the range $[0:3]$. With verbosity $0$, there
is no terminal output.

Affinity clustering options:
\texttt{\small --max-iterations         [int]  Maximum number of iterations in affinity propagation clustering .}{\small \par}
\texttt{\small --convergence\_iterations [int]   Number of iterations the set of representatives has to stay constant .}{\small \par}
\texttt{\small --damping Ratio of previous iteration entering values. Value between [0.5:1). .}{\small \par}
\texttt{\small --similarity-type Type of score used for clustering [1:5]. 1=alignment score. 2=coverage 3=sequence identity 4=E-value 5= Score per Column .}{\small \par}
\texttt{\small --preference Preference value influences the number of clusters (default=0). High values lead to more clusters. .}{\small \par}







\section{License terms}

The software is made available under the terms of the GNU General
Public License v3.0. Its contributors assume no responsibility for
errors or omissions in the software.


\end{document}
